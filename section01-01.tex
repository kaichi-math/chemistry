% !TEX root = main.tex

\parindent0pt
% \setatomsep{1em}
\setchemfig{atom sep=2em}

\testtitle{化学用語集}{[1]}

\textbf{飽和炭化水素}\quad 
\ce{C}と\ce{H}から成るから炭化水素. 飽和は\ce{H}でイッパイということ.
これ以上\ce{H}が増やせないというのは, 不飽和な\ce{CH2=CH-CH3}や\ce{CH#C-CH3}と比較して理解する.\\

\textbf{アルカン} alkane \ce{C_nH_{2n+2}} \quad 飽和炭化水素で, 鎖式のものの総称.
以下に挙げるのは, 分岐していない, 直鎖のもの.
\ce{C}の個数が名前の由来になる. \(\ce{C}\leqq10\) までは覚える.\\

\begin{tabular}{llllllll}
1. mono  & \ce{CH4}    & \ce{CH4}                & \textbf{メタン} methane   \\
2. di    & \ce{C2H6}   & \ce{CH3 - CH3}          & \textbf{エタン} ethane    \\
3. tri   & \ce{C3H8}   & \ce{CH3 - CH2 - CH3}    & \textbf{プロパン} propane \\
4. tetra & \ce{C4H10}  & \ce{CH3 - (CH2)2 - CH3} & \textbf{ブタン} butane    \\
5. penta & \ce{C5H12}  & \ce{CH3 - (CH2)3 - CH3} & ペンタン pentane \\
6. hexa  & \ce{C6H14}  & \ce{CH3 - (CH2)4 - CH3} & ヘキサン hexane  \\
7. hepta & \ce{C7H16}  & \ce{CH3 - (CH2)5 - CH3} & ヘプタン heptane \\
8. octa  & \ce{C8H18}  & \ce{CH3 - (CH2)6 - CH3} & オクタン octane  \\
9. nona  & \ce{C9H20}  & \ce{CH3 - (CH2)7 - CH3} & ノナン nonane    \\
10. deca & \ce{C10H22} & \ce{CH3 - (CH2)8 - CH3} & デカン decane    \\
\end{tabular}\\

de\underline{c}a が c であることを覚えておくと, 以下に出てくる音の変化がつかみやすい.\\

\textbf{パラフィン} paraffine は\(\ce{C}\geqq20\)のアルカンの総称.\\

\textbf{アルキル基} alkyl group \ce{- C_nH_{2n+1}} \quad アルカンの\ce{H}が一つ脱落した置換基.
以下に挙げるのは, 直鎖の端にある\ce{H}が脱落したもの.
名称は, -ane が -yl になる.\\

\begin{tabular}{llllllll}
1. mono  & \ce{- CH3}    & \ce{- CH3}        & メチル基 methyl group   \\
2. di    & \ce{- C2H5}   & \ce{- CH2-CH3}    & エチル基 ethyl group    \\
3. tri   & \ce{- C3H7}   & \ce{- (CH2)2-CH3} & プロピル基 propyl group \\
4. tetra & \ce{- C4H9}   & \ce{- (CH2)3-CH3} & ブチル基 butyl group    \\
5. penta & \ce{- C5H11}  & \ce{- (CH2)4-CH3} & ペンチル基 pentyl group \\
6. hexa  & \ce{- C6H14}  & \ce{- (CH2)5-CH3} & ヘキシル基 hexyl group  \\
7. hepta & \ce{- C7H15}  & \ce{- (CH2)6-CH3} & ヘプチル基 heptyl group \\
8. octa  & \ce{- C8H17}  & \ce{- (CH2)7-CH3} & オクチル基 octyl group  \\
9. nona  & \ce{- C9H19}  & \ce{- (CH2)8-CH3} & ノニル基 nonyl group    \\
10. deca & \ce{- C10H21} & \ce{- (CH2)9-CH3} & デシル基 decyl group    \\
\end{tabular}\\

\testtitle{化学用語集}{[2]}

\textbf{不飽和炭化水素}\quad 
\ce{C}と\ce{H}から成るから炭化水素のうち, 二重結合\ce{C=C}や三重結合\ce{C#C}を含むもの.
重結合を切れば, まだ\ce{H}を付けられるから不飽和.\\

\textbf{アルケン} alkene \ce{C_nH_{2n}} \quad 
直鎖で二重結合を一つもつ炭化水素. 当然ながら, \(\ce{C}=1\) のものはない.\\

\begin{tabular}{llllllll}
1. mono  &             &                            &                 \\
2. di    & \ce{C2H4}   & \ce{CH2=CH2}               & エテン ethene    (\textbf{エチレン} ethylene)    \\
3. tri   & \ce{C3H6}   & \ce{CH2=CH - CH3}          & プロペン propene (\textbf{プロピレン} propylene) \\
4. tetra & \ce{C4H8}   & \ce{CH2=CH - CH2 - CH3}    & ブテン butene    (ブチレン butylene)       \\
5. penta & \ce{C5H10}  & \ce{CH2=CH - (CH2)2 - CH3} & ペンテン pentene \\
6. hexa  & \ce{C6H12}  & \ce{CH2=CH - (CH2)3 - CH3} & ヘキセン hexene  \\
7. hepta & \ce{C7H14}  & \ce{CH2=CH - (CH2)4 - CH3} & ヘプテン heptene \\
8. octa  & \ce{C8H16}  & \ce{CH2=CH - (CH2)5 - CH3} & オクテン octene  \\
9. nona  & \ce{C9H18}  & \ce{CH2=CH - (CH2)6 - CH3} & ノネン nonene    \\
10. deca & \ce{C10H20} & \ce{CH2=CH - (CH2)7 - CH3} & デセン decene    \\
\end{tabular}\\

\(\ce{C}\geqq3\) のブテンからは, 二重結合の位置に自由度がある.
表中のものは, 1-ブテン(1-butene)だが, 
\begin{center}
2-ブテン(2-buten) \quad \ce{CH3-CH=CH-CH3} 
\end{center}
も考えられる.
それどころか, シス-2-ブテン(cis-2-butene)とトランス-2-ブテン(trans-2-butene)の可能性も考えられる.
\begin{center}
\chemfig{H_3C-[:-60]C(-[:-120]H)=C(-[:-60]H)-[:60]CH_3}
\qquad\chemfig{H_3C-[:-60]C(-[:-120]H)=C(-[:60]H)-[:-60]CH_3}
\end{center}
これらは\textbf{構造異性体}と呼ばれる関係にある.\\

ところで, この 1- や 2- の数だが, 不飽和炭化水素の場合, 
多重結合を含む最も長い直鎖を主鎖として, 
多重結合に割り当てられる番号が最も小さくなるように端から振るらしい.
(多重結合が増えたらどうなるのか, 考え出したらキリがない.)

\testtitle{化学用語集}{[3]}

\textbf{アルキン} alkyne \ce{C_nH_{2n-2}} \quad 
直鎖で三重結合を一つもつ炭化水素. こちらも当然ながら, \(\ce{C}=1\) のものはない.\\

\begin{tabular}{llllllll}
1. mono  &             &                        &                 \\
2. di    & \ce{C2H2}   & \ce{CH#CH}             & エチン ethyne    (\textbf{アセチレン} acetylene)    \\
3. tri   & \ce{C3H4}   & \ce{CH#C -CH3}         & プロピン propyne \\
         &             &                        & (\textbf{メチルアセチレン} methylacetylene) \\
4. tetra & \ce{C4H6}   & \ce{CH#C -CH2 - CH3}   & ブチン butyne    \\
5. penta & \ce{C5H8}   & \ce{CH#C -(CH2)2 -CH3} & ペンチン pentyne \\
6. hexa  & \ce{C6H10}  & \ce{CH#C -(CH2)3 -CH3} & ヘキシン hexyne  \\
7. hepta & \ce{C7H12}  & \ce{CH#C -(CH2)4 -CH3} & ヘプチン heptyne \\
8. octa  & \ce{C8H14}  & \ce{CH#C -(CH2)5 -CH3} & オクチン octyne  \\
9. nona  & \ce{C9H16}  & \ce{CH#C -(CH2)6 -CH3} & ノニン nonyne    \\
10. deca & \ce{C10H18} & \ce{CH#C -(CH2)7 -CH3} & デシン decyne    \\
\end{tabular}\\

アルキンについても, 末端アルキンだけではなく, 
内部アルキンを考えられる.
2-ブチン (2-butyne) 
\begin{center}
\ce{CH2 -C#C-CH2}
\end{center}
などである. \ce{C}が直線状に並ぶことを確認しておきたい.\\

alkane, alkene, alkyne は英語の音としては, アルケイン, アルキーン, アルカインになり, 
カタカナ表記とずれる. 15C前後に生じた大母音推移を実感できる現象だ.

しかし, そもそも, なんでカケキなのか. 
ドイツ語表記が Alkane, Alkene, Alkine となることから推察するに, 
アルファベット順 a, e, i なのではなかろうか. だとすれば, 四重結合や五重結合が存在し得たら, 
アルコン, アルクンになったに違いない.\\

\textbf{ビニル基} vinyl group \ce{-C=CH2} \quad
アルケニル基 (alkenyl group) \ce{-C_nH_{2n-1}}, 
アルキニル基 (alkynyl group) \ce{-C_nH_{2n-3}} のような総称もあるらしいが, 
重要なものはビニル基のみだろう.

\testtitle{化学用語集}{[4]}

\textbf{アルコール} alcohol \ce{C_nH_m-OH} \quad
炭化水素の\ce{-H}をヒドロキシ基(hydroxy group) \ce{-OH}で置換.\\

炭素数の順に,

\begin{tabular}{llllllll}
1. mono  & \ce{CH3-OH}            & \textbf{メタノール} methanol (メチルアルコール methyl alcohol)  \\
2. di    & \ce{CH3-CH2-OH}        & \textbf{エタノール} ethanol (エチルアルコール ethyl alcohol)   \\
3. tri   & \ce{CH3-CH2-CH2-OH}    & \textbf{プロパノール} propanol \\
4. tetra & \ce{CH3-(CH2)2-CH2-OH} & \textbf{ブタノール} butanol    \\
\end{tabular}\\

であり, ペンタノール以下も同様.\\
炭素数が多いものを\textbf{高級アルコール}と呼ぶが, 別におタカいおサケではない.\\

\ce{-OH}と結合した\ce{C}がいくつの\ce{H}と結合するかに注目して, 第一級から第三級に分類する.

\begin{center}
\begin{tabular}{llll}
\multicolumn{4}{l}{\textbf{第一級アルコール} primary}\\
\chemfig{H-C(-[:-90]OH)(-[:90]H)-H} &
\chemfig{H-C(-[:-90]OH)(-[:90]H)-CH_3} &
\chemfig{H-C(-[:-90]OH)(-[:90]H)-CH_2-CH_3} &
\chemfig{H-C(-[:-90]OH)(-[:90]H)-CH(-[:90]CH_3)-CH_3} \\
メタノール & エタノール & 1-プロパノール & 2-メチル-1-プロパノール \\
& & & 2-methylpropan-1-ol \\
& & & (イソブチルアルコール) \\
& & & (isobutanol) \\[4pt]
\multicolumn{3}{l}{\textbf{第二級アルコール} secondary} & \multicolumn{1}{l}{\textbf{第三級アルコール} tertiary} \\
\multicolumn{3}{l}{\chemfig{H_3C-C(-[:90]H)(-[:-90]OH)-CH_3}} 
& \multicolumn{1}{l}{\chemfig{H_3C-C(-[:90]CH_3)(-[:-90]OH)-CH_3}} \\
\multicolumn{3}{l}{2-プロパノール (イソプロパノール)} & \multicolumn{1}{l}{2-メチル-2-プロパノール} \\
\multicolumn{3}{l}{propan-2-ol (isopropanol)} & \multicolumn{1}{l}{2-methylpropan-2-ol} 
\end{tabular}
\end{center}
各構造式から, 炭素の位置を表す 1- や 2- のルールを掴むこと. 見やすい向きに描いている\ldots はず.
\vskip9pt

\ce{-OH}の個数に注目して分類するときは, 価数と呼ぶ.

\begin{center}
\begin{tabular}{llll}
\textbf{一価アルコール} & \textbf{二価アルコール} diol & \textbf{三価アルコール} triol \\
\chemfig{H-C(-[:-90]OH)(-[:90]H)-H} &
\chemfig{H-C(-[:-90]OH)(-[:90]H)-C(-[:-90]OH)(-[:90]H)-H} &
\chemfig{H-C(-[:-90]OH)(-[:90]H)-C(-[:-90]OH)(-[:90]H)-C(-[:-90]OH)(-[:90]H)-H} \\
メタノール & 1,2-エタンジオール (\textbf{エチレングリコール}) & 1,2,3-プロパントリオール (\textbf{グリセリン})\\
mathanol & ethane-1,2-diol (ethylene glycol) & propane-1,2,3-triol (glycerine) 
\end{tabular}
\end{center}

エチレングリコールはPET (ポリエチレンテレフタラート)の材料として, 
グリセリンは脂質の中心部分として重要. ここで覚えてしまおう.